\documentclass[%
	a4paper,
	oneside,
	listof=numbered,
	parskip=half,
	headsepline=true,
	footsepline=false,
	normalheadings,
	0.7headlines,
	headexclude,
	]{scrbook}

\usepackage{scrlayer-scrpage}

\usepackage[english]{babel}
\usepackage{csquotes}
\usepackage[fixlanguage]{babelbib}
\selectbiblanguage{english}

\usepackage{graphicx}
\usepackage{xcolor}
\usepackage{xltabular}
\usepackage{wrapfig}
\usepackage{adjustbox}
\usepackage{rest-api}
% listings package requires the scrhack to
% work properly and ommit a warning message
\usepackage{scrhack}
\usepackage{listings}

\usepackage{tikz}
\usetikzlibrary{shapes,positioning}

\usepackage{typearea}
\areaset[1cm]{15cm}{23cm} % selbst setzen

% Style-Definition für die Code-Listings
\lstdefinestyle{mystyle}{
     basicstyle=\ttfamily\footnotesize,
     breakatwhitespace=false,
     breaklines=true,
     captionpos=t,
     frame=lines,
     keepspaces=true,
     numbers=left,
     numbersep=5pt,
     showspaces=false,
     showstringspaces=false,
     showtabs=false,
     tabsize=2,
     columns=fullflexible,
}
\lstset{style=mystyle}

\definecolor{gray}{rgb}{0.4,0.4,0.4}
\definecolor{darkblue}{rgb}{0.0,0.0,0.6}
\definecolor{cyan}{rgb}{0.0,0.6,0.6}

\lstdefinelanguage{XML}
{
  morestring=[b]",
  morestring=[s]{>}{<},
  morecomment=[s]{<?}{?>},
  stringstyle=\color{black},
  identifierstyle=\color{darkblue},
  keywordstyle=\color{cyan},
  morekeywords={xmlns,version,type,name,id,rid,index}% list your attributes here
}

\titlehead{%
	\adjustimage{%
		width=\paperwidth,
		pagecenter,
		valign=B,
		raise=20ex,
		}{ECAD-Export-Headline}
	}

\pagestyle{scrheadings}

\begin{document}
% --------------------------------------------------------------
\title{ECAD Interchange Format}
\subtitle{Phoenix Contact CLIP PROJECT Version 6.5}

\subject{ALX-Development}
\author{Dipl.-Ing. (FH) Alexander Thiel M.Sc.}

% Mit diesem Kommando kann das Datum festgelegt werden. Das ist sinnvoll, wenn dieser Bericht fertig ist und sich das Datum nicht immer, wenn er neu generiert wird wieder ändern soll.
%\date{30. Februar 2022}
%\publishers{...}

\maketitle
\tableofcontents
% --------------------------------------------------------------

\chapter{Introduction}

This document is not an official documentation of the ECAD Exchange file format used by Phoenix Contacts
CLIP PROJECT software, it is based upon researches and reengineering of exchange files by myself. So there is no warrenty, that this documentation is complete or correct in all cases. 

The generic ECAD data exchange format provides full exchange of all relevant information based on a XML structured file. The objective of this document is to describe this XML format.

\section{XML node structure}

The ECAD exchange file format is XML based. One ECAD file contains only one project with unlimited number of mounting rails and terminal strips.

The structure starts with the root object \verb|<pbf>| and it contains exact one single project object \verb|<o type='clipprj.project'>|. The project object may contain multiple other objects which are surrounded by the \verb|<ol>| (object list) tag. Based on this structure a project according the hierarchy as shown on the diagram in figure \ref{pict:structure_erm} can be created.

\tikzset{branode/.style={midway,font=\scriptsize}}

\begin{figure}[!ht]
\centering
\resizebox{\textwidth}{!}{
\begin{tikzpicture}[>=stealth]

\node[](pbf){\begin{tabular}{|c|} \hline
	\textbf{Root Object}\\ \hline
	<pbf>\\ \hline
	\end{tabular}};

\node[below=of pbf](prj){\begin{tabular}{|c|} \hline
	\textbf{Project Object}\\ \hline
	<o type='clipprj.project'>\\ \hline
	\end{tabular}};

\node[below left=of prj](loc){\begin{tabular}{|c|} \hline
	\textbf{Location Object}\\ \hline
	<o type='clipprj.location'>\\ \hline
	\end{tabular}};

\node[below right=of prj](ref){\begin{tabular}{|c|} \hline
	\textbf{Table of References}\\ \hline
	<refs>, <ref>\\ \hline
	\end{tabular}};

\node[below=3cm of prj](rail){\begin{tabular}{|c|} \hline
	\textbf{Rail Objects}\\ \hline
	<o type='clipprj.rail'>\\ \hline
	\end{tabular}};

\node[below left=of rail](onrail){\begin{tabular}{|c|} \hline
	\textbf{On-rail Objects}\\ \hline
	<o type='clipprj.terminal'>\\
	<o type='clipprj.plc'>\\ \hline
	\end{tabular}};

\node[below right=of rail](acc){\begin{tabular}{|c|} \hline
	\textbf{Accessory Objects}\\ \hline
	<o type='clipprj.accessory'>\\
	<o type='clipprj.bridge'>\\ \hline
	\end{tabular}};

\draw[->](pbf)--(prj)node[branode,left=1mm]{[$1$]};
\draw[->](prj)--(rail)node[branode,left=1mm]{[$0..n$]};
\draw[->](prj)--(loc)node[branode,left,anchor=south east]{[$0..n$]};
\draw[->](loc)--(rail)node[branode,left,anchor=south west]{[$0..n$]};
\draw[->](rail)--(onrail)node[branode,left,anchor=south east]{[$0..n$]};
\draw[->](rail)--(acc)node[branode,left,anchor=south west]{[$0..n$]};

\end{tikzpicture}}
\caption{Structure of the ECAD file}
\label{pict:structure_erm}
\end{figure}

An example header of the ECAD file is shown in listing \ref{code:xml_example_header}.

\lstinputlisting[%
	label={code:xml_example_header},
	caption={Example header of the ECAD file},
	language=XML,
	firstline=1,
	lastline=10,
]{ECAD-Export-ExampleProject.xml}

The ECAD file format bases on limited tag elements. Below there is full list of allowed tag elements.

\subsection{Tag description}

\subsubsection{Root object}

Root object of the XML file which provide information about version, format and schema.

\begin{description}
	\item[Tag] \verb|<pbf>|
	\item[Attributes]:
	
	\begin{description}
		\item[version] version of ECAD file. Current version is 2.0
		\item[format] defines namespace for the file. Must be “clipprj”
	\end{description}
\end{description}

\subsubsection{List of objects}

Contain collection of objects.

\begin{description}
	\item[Tag] \verb|<ol>|
\end{description}

\subsubsection{Objects}

Creates object of given type. Object may contain other objects (list of objects), properties (list of properties) and reference object (list of reference objects)

\begin{description}
	\item[Tag] \verb|<o>|
	\item[Attributes]:
	
	\begin{description}
		\item[type] type name of the object
		\item[name] name of the object
		\item[id] unique id of the object
	\end{description}
\end{description}

Name attribute is ignored if name property is added in list of properties additionally.

\subsubsection{List of properties}

Contains collection of properties

\begin{description}
	\item[Tag] \verb|<pl>|
\end{description}

\subsubsection{Property}

Represent one property of the object with the value.

\begin{description}
	\item[Tag] \verb|<p>|
	\item[Attributes]:
	
	\begin{description}
		\item[name] name of property
	\end{description}
\end{description}

Example: \verb|<p name="clipprj.articleNumber">0801681</p>|

\subsubsection{List of reference objects}

Contain list of reference objects

\begin{description}
	\item[Tag] \verb|<rl>|
\end{description}

\subsubsection{Reference object}

Create reference object that refers to the table of references by rid value. It allows to create relation between 2 objects in ECAD file (see also example listing \ref{code:xml_example_reference}.

\begin{description}
	\item[Tag] \verb|<r>|
	\item[Attributes]:
	
	\begin{description}
		\item[rid] reference id that refers to table of references entries.
		\item[name] name of reference object. Can be omitted.
	\end{description}
\end{description}

\begin{lstlisting}[%
	label={code:xml_example_reference},
	caption={Example of referencing objects},
	language=XML,
]
<o type="clipprj.terminal" id="1">
...
	<rl>
		<r name="accessory" rid="4"/>
	</rl>
</o>

<o type="clipprj.accessory" id="2">
...
	<rl>
		<r name="accessory" rid="4"/>
	</rl>
</o>

<refs>
	<ref id="4" oid="2"/>
</refs>
\end{lstlisting}

\subsubsection{Table of references}

Contain collection of references.

\begin{description}
	\item[Tag] \verb|<refs>|
\end{description}

\subsubsection{Reference}

Represent one reference to any object. Together with reference object, it allows to create relation between 2 objects in ECAD file.

\begin{description}
	\item[Tag] \verb|<ref>|
	\item[Attributes]:
	
	\begin{description}
		\item[id] id of reference
		\item[oid] id of referenced object
	\end{description}
\end{description}

\chapter{Specification}

\section{General XML definitions}

\begin{itemize}
	\item The ECAD uses the clipprj namespace and it has to be specified in it’s \verb|<pbf>| format attribute: \verb|<pbf format="clipprj" …| 
	\item clipprj namespace has to be use in every <p>name and <o>type attribute. 
	\item  First \verb|<o>| object has to be defined as project with attribute \verb|type=clipprj.project|
	\verb|<o id="0" name="ClipProject1" type="clipprj.project">|
	\item Attribute name is project name. 
	\item In a single file only one project can be defined. 
	\item Project can include any kind of locations and rails.
\end{itemize}

\section{XML tags}

The ECAD has a very limited number of used tags. These are:

\begin{itemize}
	\item pbf [version, format] 
	\item o [name, type, id] 
	\item ol 
	\item pl 
	\item p [name] 
	\item rl 
	\item r [index, name, rid] 
	\item refs 
	\item ref [id, oid] 
\end{itemize}

\section{Tag attributes}

The strings in brackets indicate the allowed attributes for the given tags. 

\begin{description}
	\item[<pbf>] is the root object of the file and contain a version number, format name and the schema information. 
	\item[<o>] is object and has a name, a type and id as xml attributes. Every id number has to be unique in a project and it identifies every unambiguous <o> element. id attributes are used in references. 
	\item[<ol>] is object list. Object list element can contain any number of <o> child objects, thus allowing a tree structure of objects to be create. 
	\item[<pl>] is attribute list. In \verb|<pl>| it is possibility to store any number of \verb|<p>| elements. 
	\item[<p>] is attribute and has a name as xml attributes and a value as xml text. 
	\item[<rl>] is reference list. In \verb|<rl>| there may be any number of <r> elements. The rid is an indirect reference to another object in the tree via the \verb|<ref>| table. 
	\item[<r>] is reference object. \verb|<r>| has a name, a index and a rid as xml attributes. For every \verb|<rl>| index attribute has to be a unique value. 
	\item[<refs>] is table of reference. In <refs> there may be any number of <ref> elements. 
	\item[<ref>] is reference with two xml attributes: a id and oid. \verb|<ref>| is only needed if the object structure uses references. For each object referenced there will be one \verb|<ref>| object which maps the rid to the corresponding oid of the object in the tree. 
\end{description}

\section{Object types}

CLIP PROJECT ECAD defined following object types ( <o>type ): 
 
\subsection{clipprj.project}
 
clipprj.project – is project.

There can be only one project defined in file. Project contains locations (optionally) and rails.

The following attributes may be used:

\begin{description}
	\item[type] must be clipprj.project
	\item[name] name of the project
	\item[id] unique id of the object
\end{description}
 
Project describes the following \verb|<p>| attributes: \emph{no properties}

\subsection{clipprj.location}
 
clipprj.location – is location. 

Define location. There can be unlimited number of location objects which create hierarchical structure. Location contains other locations (optionally) and rails.

The following attributes may be used:

\begin{description}
	\item[type] must be clipprj.location
	\item[name] name of the location
	\item[id] unique id of the object
\end{description}

Location describes the following \verb|<p>| attributes: \emph{no properties}

\subsection{clipprj.rail}

Define rail to assembly terminal strip. There can be unlimited number of rails created directly under project object or under location. Rail contains on-rail elements like terminals. Default article number for rail if not specified is: 0801733 and length 2000 mm.

The following attributes may be used:

\begin{description}
	\item[type] must be clipprj.rail
	\item[name] name of the rail
	\item[id] unique id of the object
\end{description}

Rail describes the following \verb|<p>| attributes: 

\begin{description}
	\item[clipprj.articleNumber] article number, default 0801733 
	\item[clipprj.name] name 
	\item[clipprj.length] length [mm], default 2000 
	\item[clipprj.edgeOfCut] edge of cut [mm], default 0. Allowed range of value is between –5 and 5. 
	\item[clipprj.spaceLeft] space between left edge of the rail and first element on the rail 
	\item[clipprj.comment] comment 
	\item[clipprj.drillHoles] list of holes. Every hole use four fields separated by semicolon as: distance from left edge of the rail;x – size;y – size;type of the hole e.g. 200;2;3;0;230;2;2;1 ... Type of the hole is 0 for hole, 1 for nut and 2 for screw. 
	\item[clipprj.drillType] 0 for every type of holes or 1 for types defined in ClipProject. 
	\item[clipprj.endCap] End cap article number if have to be add. 
\end{description}

\subsection{clipprj.shoppingCart} 

Define a shopping cart. There can be unlimited number of shopping carts created directly under project object or under location. Shopping cart may contain any product but there is no links, no relations or logic.

The following attributes may be used:

\begin{description}
	\item[type] must be clipprj.shoppingCart
	\item[name] name of the shopping cart
	\item[id] unique id of the object
\end{description}

Shopping cart describe following <p> attributes: 

\begin{description}
	\item[clipprj.articleNumber] article number 
	\item[clipprj.name] name 
	\item[clipprj.customer] customer 
\end{description}
 
Shopping cart in contrast to the Rail must have flat structure of elements. All elements have to lie directly on the Shopping cart (elements as well as extras). It follows that elements lie on the Shopping cart can’t have any references to extras (e.g. bridge). 

\subsection{clipprj.terminal} 
 
Terminal describes the following \verb|<p>| attributes: 

\begin{description}
	\item[clipprj.articleNumber] article number 
	\item[clipprj.rotated] 1 if rotated otherwise 0 (rotate 180°) 
	\item[clipprj.comment] comment 
	\item[clipprj.closed] use for some terminals, 0 for open or 1 for close. 
	\item[clipprj.description] list of descriptions. Every description use seven fields separated by semicolon as: Level;x-coord;y-coord;description;direction;height of text;align;color x and y are coordinates in mm.
	Description is to display text.\\
	Direction text allowed following values: 
	\verb|L| – left, \verb|R| – right, \verb|U| – up, \verb|D| – down.\\
	Height is font size.\\
	Align is: 0 for align to left or 1 for center text.\\
	Colors: 0 - White, 1 - Red, 2 - Blue, 3 - Yellow, 4 - Green, 5 - Violet, 6 - Orange. e.g. \verb|O1;3.10;2.20;txt;R;3.00;1;0...|\\
	It is possible to omit the x and y coordinates. In this case every next level description will be add to next occurrence of the same level from DB coordinates. e.g. \verb|O1;;;txt;R;3.00;1;0...|
	\item[clipprj.ECADdescription] can be used as alternative to clipprj.description. Every description use two fields separate by semicolon as: Level;description and levels are represent by numbers (e.g. 1 instead of O1) 
	\item[clipprj.connectionId] id use by bridges in clipprj.connectionpoints attribute. 
	\item[clipprj.objId] external ID (use by external programs to recognize element) 
	\item[clipprj.doNotCorrect] 1 if auto-correction should ignore this element during check otherwise 0 
\end{description}

\subsection{clipprj.svp}
 
\verb|svp| (Surge voltage protection) describes the following \verb|<p>| attributes: 

\begin{description}
	\item[clipprj.articleNumber] article number 
	\item[clipprj.rotated] 1 if rotated, otherwise 0. 
	\item[clipprj.comment] comment 
	\item[clipprj.description] list of descriptions (see Terminal) 
	\item[clipprj.ECADdescription] list of descriptions (see Terminal) 
	\item[clipprj.doNotCorrect] 1 if auto-correction should ignore this element during check otherwise 0
	\item[clipprj.objId] external ID (use by external programs to recognize element)
\end{description}

\subsection{clipprj.safetySwitch}
 
Safety switch describes the following \verb|<p>| attributes: 

\begin{description}
	\item[clipprj.articleNumber] article number
	\item[ clipprj.rotated] 1 if rotated, otherwise 0
	\item[clipprj.comment] comment
	\item[clipprj.description] list of descriptions (see Terminal)
	\item[clipprj.ECADdescription] list of descriptions (see Terminal)
	\item[clipprj.doNotCorrect] 1 if auto-correction should ignore this element during check otherwise 0
	\item[clipprj.objId] external ID (use by external programs to recognize element)
\end{description}

\subsection{clipprj.automation}
 
Automation describes the following \verb|<p>| attributes: 

\begin{description}
	\item[clipprj.articleNumber] article number 
	\item[clipprj.rotated] 1 if rotated, otherwise 0. 
	\item[clipprj.comment] comment 
	\item[clipprj.description] list of descriptions (see Terminal) 
	\item[clipprj.ECADdescription] list of descriptions (see Terminal) 
	\item[clipprj.doNotCorrect] 1 if auto-correction should ignore this element during check otherwise 0 
	\item[clipprj.objId] external ID (use by external programs to recognize element)
\end{description}

\subsection{clipprj.socket}
 
Socket describes the following \verb|<p>| attributes: 

\begin{description}
	\item[clipprj.articleNumber] article number 
	\item[clipprj.description] list of descriptions (see Terminal). 
	\item[ clipprj.ECADdescription] list of descriptions (see Terminal) 
	\item[clipprj.rotated] 1 if rotated, otherwise 0. 
	\item[clipprj.comment] comment 
	\item[clipprj.doNotCorrect] 1 if auto-correction should ignore this element during check otherwise 0 
	\item[clipprj.objId] external ID (use by external programs to recognize element) 
\end{description}

\subsection{clipprj.powerSupply}
 
Power Supply describes the following \verb|<p>| attributes: 

\begin{description}
	\item[clipprj.articleNumber] article number
	\item[clipprj.rotated] 1 if rotated, otherwise 0
	\item[clipprj.rotated90] 1 if rotated, otherwise 0 (rotate $90^\circ$) 
	\item[clipprj.comment] comment
	\item[clipprj.doNotCorrect] 1 if auto-correction should ignore this element during check otherwise 0
	\item[clipprj.objId] external ID (use by external programs to recognize element) Rotate combination allow rotate element $90^\circ$, $180^\circ$ or $270^\circ$
\end{description}

\subsection{clipprj.plcRelais}
 
PLC Relais describes the following \verb|<p>| attributes: 

\begin{description}
	\item[clipprj.articleNumber] article number 
	\item[clipprj.rotated] 1 if rotated, otherwise 0. 
	\item[clipprj.comment] comment 
	\item[clipprj.description] list of descriptions (see Terminal). 
	\item[clipprj.ECADdescription] list of descriptions (see Terminal) 
	\item[clipprj.optionText] text to be displayed in properties dialog. 
	\item[clipprj.doNotCorrect] 1 if auto-correction should ignore this element during check otherwise 0 
	\item[clipprj.objId] external ID (use by external programs to recognize element)
\end{description}

\subsection{clipprj.blackbox}
 
Black box describes the following \verb|<p>| attributes: 

\begin{description}
	\item[clipprj.articleNumber] article number 
	\item[clipprj.rotated] 1 if rotated, otherwise 0. 
	\item[clipprj.comment] comment 
	\item[clipprj.width] width [mm], default 20 mm 
	\item[clipprj.height] height [mm], default 30 mm 
	\item[clipprj.depth] depth [mm], default 40 mm 
	\item[clipprj.insertPoint] Y-coordinate [mm], default 2.5 mm 
	\item[clipprj.bmpFile] bitmap element for Graphics viewer, default \verb|BB_standard.bmp|
	\item[clipprj.objId] external ID (use by external programs to recognize element) 
\end{description}

\subsection{clipprj.cabeling}
 
Cabeling describes the following \verb|<p>| attributes: 

\begin{description}
	\item[clipprj.articleNumber] article number 
	\item[clipprj.rotated] 1 if rotated, otherwise 0. 
	\item[clipprj.comment] comment 
	\item[clipprj.doNotCorrect] 1 if auto-correction should ignore this element during check otherwise 0 
	\item[clipprj.objId] external ID (use by external programs to recognize element) 
\end{description}

\subsection{clipprj.signalConverter}
 
Signal Converter describes the following \verb|<p>| attributes: 

\begin{description}
	\item[clipprj.articleNumber] article number 
	\item[clipprj.rotated] 1 if rotated, otherwise 0. 
	\item[clipprj.comment] comment 
	\item[clipprj.doNotCorrect] 1 if auto-correction should ignore this element during check otherwise 0 
	\item[clipprj.objId] external ID (use by external programs to recognize element) 
\end{description}

\subsection{clipprj.safetyRelais}
 
Safety Relais describes the following \verb|<p>| attributes: 

\begin{description}
	\item[clipprj.articleNumber] article number 
	\item[clipprj.rotated] 1 if rotated, otherwise 0. 
	\item[clipprj.comment] comment 
	\item[clipprj.doNotCorrect] 1 if auto-correction should ignore this element during check otherwise 0 
	\item[clipprj.objId] external ID (use by external programs to recognize element) 
\end{description}

\subsection{clipprj.generic}
 
Generic is a generic type. A Generic type can be used when given article number is existing in CLIP PROJECT database (article number is known by CLIP PROJECT). More attributes can be either applied or ignored in dependence on type of the element.

\subsection{clipprj.endBracket}
 
End Bracket describes the following \verb|<p>| attributes: 

\begin{description}
	\item[clipprj.articleNumber] article number 
	\item[clipprj.rotated] 1 if rotated, otherwise 0. 
	\item[clipprj.description] list of descriptions (see Terminal). 
	\item[clipprj.ECADdescription] list of descriptions (see Terminal). 
	\item[clipprj.comment] comment 
	\item[clipprj.doNotCorrect] 1 if auto-correction should ignore this element during check otherwise 0 
	\item[clipprj.objId] external ID (use by external programs to recognize element) 
\end{description}

\subsection{clipprj.blockingSurface}
 
Blocking Surface describes the following \verb|<p>| attributes: 

\begin{description}
	\item[clipprj.width] is width [mm] 
	\item[clipprj.comment] comment 
	\item[clipprj.doNotCorrect] 1 if auto-correction should ignore this element during check otherwise 0 
	\item[clipprj.objId] external ID (use by external programs to recognize element) 
\end{description}

\subsection{clipprj.supportTrestle}
 
Support Trestle describes the following \verb|<p>| attributes: 

\begin{description}
	\item[clipprj.articleNumber] article number 
	\item[clipprj.rotated] 1 if rotated, otherwise 0. 
	\item[clipprj.comment] comment 
	\item[clipprj.doNotCorrect] 1 if auto-correction should ignore this element during check otherwise 0 
	\item[clipprj.objId] external ID (use by external programs to recognize element) 
\end{description}

\subsection{clipprj.partitionPlate}
 
Partition Plate describes the following \verb|<p>| attributes: 

\begin{description}
	\item[clipprj.articleNumber] article number 
	\item[clipprj.rotated] 1 if rotated, otherwise 0. 
	\item[clipprj.comment] comment 
	\item[clipprj.doNotCorrect] 1 if auto-correction should ignore this element during check otherwise 0 
	\item[clipprj.objId] external ID (use by external programs to recognize element) 
\end{description}

\subsection{clipprj.endCover}
 
End Cover describes the following \verb|<p>| attributes: 

\begin{description}
	\item[clipprj.articleNumber] article number 
	\item[clipprj.rotated] 1 if rotated, otherwise 0. 
	\item[clipprj.comment] comment 
	\item[clipprj.doNotCorrect] 1 if auto-correction should ignore this element during check otherwise 0 
	\item[clipprj.objId] external ID (use by external programs to recognize element) 
\end{description}

\subsection{clipprj.distanceCover}
 
Distance Cover describes the following \verb|<p>| attributes: 

\begin{description}
	\item[clipprj.articleNumber] article number 
	\item[clipprj.rotated] 1 if rotated, otherwise 0. 
	\item[clipprj.comment] comment 
	\item[clipprj.doNotCorrect] 1 if auto-correction should ignore this element during check otherwise 0 
	\item[clipprj.objId] external ID (use by external programs to recognize element) 
\end{description}

\subsection{clipprj.separatingPlate}
 
Separating Plate describes the following \verb|<p>| attributes: 

\begin{description}
	\item[clipprj.articleNumber] article number 
	\item[clipprj.rotated] 1 if rotated, otherwise 0. 
	\item[clipprj.comment] comment 
	\item[clipprj.doNotCorrect] 1 if auto-correction should ignore this element during check otherwise 0 
	\item[clipprj.objId] external ID (use by external programs to recognize element) 
\end{description}

\subsection{clipprj.spacerPlate}
 
Spacer Plate describes the following \verb|<p>| attributes: 

\begin{description}
	\item[clipprj.articleNumber] article number 
	\item[clipprj.rotated] 1 if rotated, otherwise 0. 
	\item[clipprj.comment] comment 
	\item[clipprj.doNotCorrect] 1 if auto-correction should ignore this element during check otherwise 0 
	\item[clipprj.objId] external ID (use by external programs to recognize element) 
\end{description}

\section{Object types \glqq{}Extras\grqq{}} 
 
Second group of objects are Extras. Extras are objects that exists only with elements mounted on the rail – plus they have to be related to them. Extras with related elements have to be in the same location but not on the rail. 

\subsection{clipprj.bridge}
 
Bridge describes the following \verb|<p>| attributes: 

\begin{description}
	\item[clipprj.articleNumber] article number 
	\item[clipprj.comment] comment 
	\item[clipprj.closed] 1 if closed otherwise 0 
	\item[clipprj.coordinates] list of coordinates. Every coordinate use two fields separated by semicolon as: x – coordinate;y – coordinate. For Bridge is possibility to set terminal levels instead of coordinates (e.g. \verb|O1;U1|). 
	\item[clipprj.ECADcoordinates] can be used as alternative to clipprj.coordinates. Attribute can has only one terminal level for all connected terminals. In this case levels are represent by number (e.g. 1 instead of O1). 
	\item[clipprj.connectionpoints] list of terminals connectionId attribute values. Can be used as alternative to reference mechanism to indicate connected terminals. 
	\item[clipprj.objId] external ID (use by external programs to recognize element) 
\end{description}

\subsection{clipprj.neutralbusbar}
 
Neutral busbar describes the following \verb|<p>| attributes: 

\begin{description}
	\item[clipprj.articleNumber] article number 
	\item[clipprj.comment] comment 
	\item[clipprj.objId] external ID (use by external programs to recognize element) 
\end{description}

\subsection{clipprj.plug}
 
Plug (Module plug or Modular test plug) describes the following \verb|<p>| attributes: 

\begin{description}
	\item[clipprj.articleNumber] article number for first plug part. 
	\item[clipprj.constructionArticles] article numbers for next plug parts separated by semicolon. 
	\item[clipprj.coordinates] list of y – coordinates [mm] for every plug part coordinate
	\item[clipprj.name] module name 
	\item[clipprj.description] list of descriptions (see Terminal) 
	\item[clipprj.ECADdescription] list of descriptions (see Terminal) 
	\item[clipprj.comment] comment 
	\item[clipprj.objId] external ID (use by external programs to recognize element) 
\end{description}

\subsection{clipprj.safetySwitchPlug}
 
Safety switch plug describes the following \verb|<p>| attributes: 

\begin{description}
	\item[clipprj.articleNumber] article number 
	\item[clipprj.comment] comment 
	\item[clipprj.coordinates] y – coordinate [mm] 
	\item[clipprj.levelPosition] position plug on the basis element,
	\begin{itemize}
		\item 0 for plug inserted from left edge of basis or
		\item 1 for plug inserted on second position of basis.
	\end{itemize} 
	\item[clipprj.objId] external ID (use by external programs to recognize element) 
\end{description}

\subsection{clipprj.svpPlug}

Defines plug for Surge Voltage Protection element.

Surge voltage protection plug describes the following \verb|<p>| attributes: 

\begin{description}
	\item[clipprj.articleNumber] article number 
	\item[clipprj.comment] comment 
	\item[clipprj.coordinates] x and y coordinates [mm] separated by semicolon. Format is: \verb|x;y|
	\item[clipprj.objId] external ID (use by external programs to recognize element)
\end{description}

\subsection{clipprj.plcPlug}

Defines plug for PLC element.

PLC Relais plug describes the following \verb|<p>| attributes: 

\begin{description}
	\item[clipprj.articleNumber] article number 
	\item[clipprj.comment] comment 
	\item[clipprj.coordinates] x and y coordinates [mm] separated by semicolon. Format is: \verb|x;y|
	\item[clipprj.objId] external ID (use by external programs to recognize element)
\end{description}

\subsection{clipprj.cable}

Defines cable for Automation element.

Cable describes the following \verb|<p>| attributes: 

\begin{description}
	\item[clipprj.articleNumber] article number 
	\item[clipprj.comment] comment 
	\item[clipprj.coordinates] x and y coordinates [mm] separated by semicolon. Format is: \verb|x;y| 
	\item[clipprj.objId] external ID (use by external programs to recognize element)
\end{description}

\subsection{clipprj.accessory}
 
Accessory describes the following \verb|<p>| attributes: 

\begin{description}
	\item[clipprj.articleNumber] article number 
	\item[clipprj.comment] comment 
	\item[clipprj.coordinates] x and y coordinates [mm] separated by semicolon. Format is: \verb|x;y| 
	\item[clipprj.description] list of descriptions (see Terminal) 
	\item[clipprj.ECADdescription] list of descriptions (see Terminal) 
	\item[clipprj.objId] external ID (use by external programs to recognize element) 
\end{description}

\verb|clipprj.accessory| is used in references \verb|<r>| name to indicate reference element. 

\section{Object properties}

\subsection{Article number}

\begin{description}
	\item[Name] clipprj.articleNumber
	\item[Scope] General
\end{description}

\subsection{Rotation}

\begin{description}
	\item[Name] clipprj.rotated
	\item[Scope] General
	\item[Valid values]
	\begin{description}
		\item[1] element is flipped on rail vertically
		\item[0] element is not flipped vertically
	\end{description}
\end{description}

\subsection{Rotation horizontal}

\begin{description}
	\item[Name] clipprj.rotated90
	\item[Scope] General
	\item[Valid values]
	\begin{description}
		\item[1] element is flipped on rail horizontally
		\item[0] element is not flipped horizontally
	\end{description}
\end{description}

\emph{Note:} This property is suitable only for some elements

\subsection{Comment}

Add comment for element.

\begin{description}
	\item[Name] clipprj.comment
	\item[Scope] General
\end{description}

\subsection{Description}

\begin{description}
	\item[Name] clipprj.description
	\item[Scope] General
\end{description}

Add descriptions for elements displayed as labels. Best matching will be used for product data in CLIPPROJECT database. There may be several description and each one use 8 fields separated by semicolons:

\verb|level;x-coord;y-coord;description;direction;height;align;color;|

Where:

\begin{description}
	\item[level] id of description
	\item[x-coord, y-coord] position of description label on device in mm related to top-left corner
	\item[description] label text.
	\item[direction] text rotation, valid values:
	\begin{description}
		\item[L] left ($0^\circ$)
		\item[R] right ($180^\circ$)
		\item[U] up ($90^\circ$)
		\item[D] down ($270^\circ$)
	\end{description}
	\item[height] font size.
	\item[align] text align, valid values:
	\begin{description}
		\item[0] left
		\item[1] center
	\end{description}
	\item[color] label color (background of the text), valid values:
	\begin{description}
		\item[0] white
		\item[1] red
		\item[2] blue
		\item[3] yellow
		\item[4] green
		\item[5] violet
		\item[6] orange
	\end{description}
	Example: \verb|O1;3.10;2.20;L1;R;3.00;1;0;O2;3.10;2.20;L2;R;3.00;1;0;|
	Definition of 2 descriptions
	It is allowed to omit the x and y coordinates. In such case coordinates will be read from database for current product.
	Example: \verb|O1;;;txt;R;3.00;1;0;|
\end{description}

\subsection{ECAD description}

\begin{description}
	\item[Name] clipprj.ECADdescription
	\item[Scope] General
\end{description}

Can be used as alternative to clipprj.description. Every description use two fields separate by semicolon as: \verb|level;description|

Where:

\begin{description}
	\item[level] id of description
	\item[description] label text.
\end{description}

\subsection{Object Id}

External ID, to be used by external programs to identify object

\begin{description}
	\item[Name] clipprj.objId
	\item[Scope] General
\end{description}

\subsection{Do-Not-Correct}

Information for auto-correction that object should not be modified or removed.

\begin{description}
	\item[Name] clipprj.doNotCorrect
	\item[Scope] General
\end{description}

Valid values:

\begin{description}
	\item[1] auto-correction should ignore this element
	\item[0] default behavior
\end{description}

\subsection{Element Width}

Width of the element in mm. Width makes terminal strip longer.\\

\begin{description}
	\item[Name] clipprj.width
	\item[Scope] General
\end{description}

\subsection{Element Height}

Hegiht of the element in mm.

\begin{description}
	\item[Name] clipprj.height
	\item[Scope] General
\end{description}

\subsection{Element Depth}

Depth of the element in mm.

\begin{description}
	\item[Name] clipprj.depth
	\item[Scope] General
\end{description}

\subsection{Mounting 2D bitmap}

Bitmap file name to be used for black box object. It is displayed in 2D-Editor. File must exist in Graphics folder.

\begin{description}
	\item[Name] clipprj.bmpFile
	\item[Scope] General
\end{description}

\subsection{Vertical inserting point}

Inserting point on rail in mm

\begin{description}
	\item[Name] clipprj.insertPoint
	\item[Scope] General
\end{description}

\subsection{Name}

Name of the object. Used also as a label description if there is no Description property.

\begin{description}
	\item[Name] clipprj.name
	\item[Scope] General
\end{description}

\subsection{Potential}

Define potential type of terminal. Can be empty.

\begin{description}
	\item[Name] clipprj.potentialType
	\item[Scope] Terminal
\end{description}

Valid values:

\begin{description}
	\item[Other]
	\item[N]
	\item[PE]
	\item[SH]
\end{description}

\subsection{Connection points}

Number of connection points defined by terminal function. Can be empty.

\begin{description}
	\item[Name] clipprj.connectionPoints
	\item[Scope] Terminal
\end{description}

Valid values:

\begin{description}
	\item[0-n] Numbers only
\end{description}

\subsection{Minimum required connection points}

Number of connection points from schematic. Can be empty.\\

\begin{description}
	\item[Name] clipprj.reqConnectionPoints
	\item[Scope] Terminal
\end{description}

Valid values:

\begin{description}
	\item[0-n] Numbers only
\end{description}

\subsection{Cross section}

Cross section for terminal function in $mm^2$

\begin{description}
	\item[Name] clipprj.crossSection
	\item[Scope] Terminal
\end{description}

\subsection{Wire section required}

Wire cross section required from schematic in $mm^2$

\begin{description}
	\item[Name] clipprj. maxWireSection
	\item[Scope] Terminal
\end{description}

\subsection{Terminal category}

Terminal function type. Can be empty. Valid values: (numbers only 0-5).

\begin{description}
	\item[Name] clipprj.terminalFunction
	\item[Scope] Terminal
\end{description}

Valid values:

\begin{description}
	\item[0] Feed-through terminal
	\item[1] Isolating terminal
	\item[2] Switching terminal
	\item[3] Diode terminal
	\item[4] Fused terminal
	\item[5] Resistance terminal
\end{description}

\subsection{Terminal with LED}

Has terminal LED or not. Can be empty. Valid values: (numbers only).\\

\begin{description}
	\item[Name] clipprj.hasLED
	\item[Scope] Terminal
\end{description}

Valid values:

\begin{description}
	\item[0] no
	\item[1] yes
	\item[-1] not specified
\end{description}

\subsection{Saddle jumper}

Has terminal saddle jumper or not. Can be empty. Valid values: (numbers only).

\begin{description}
	\item[Name] clipprj.hasSaddleJumper
	\item[Scope] Terminal
\end{description}

Valid values:

\begin{description}
	\item[0] no
	\item[1] yes
	\item[-1] not specified
\end{description}

\subsection{Level}

Number of levels for terminal. Valid values: (numbers only) 1-n.

\begin{description}
	\item[Name] clipprj.level
	\item[Scope] Terminal
\end{description}

\subsection{Terminal open/closed}

Has terminal open right side. Can be empty. Valid values: (numbers only).

\begin{description}
	\item[Name] clipprj.closed
	\item[Scope] Terminal
\end{description}

Valid values:

\begin{description}
	\item[0] open
	\item[1] close
\end{description}

\subsection{Bridge connection id}

Id used by bridges in clipprj.connectionpoints attribute.

\begin{description}
	\item[Name] clipprj.connectionId
	\item[Scope] Terminal
\end{description}

\subsection{Rail length}

Length of the rail in mm. Default value is 2000 mm.

\begin{description}
	\item[Name] clipprj.length
	\item[Scope] Rail
\end{description}

\subsection{Cut Edge}

Edge of cut in mm. Default value is 0 mm.

\begin{description}
	\item[Name] clipprj.edgeOfCut
	\item[Scope] Rail
\end{description}

\subsection{Space left}

Space between left edge of the rail and first element.

\begin{description}
	\item[Name] clipprj.spaceLeft
	\item[Scope] Rail
\end{description}

\subsection{List of holes}

\begin{description}
	\item[Name] clipprj.drillHoles
	\item[Scope] Rail
\end{description}

Define holes to drill in not perforated rail. It should not be used for perforated rails. Each one hole use 4 fields separated by semicolons: \verb|distance;x-size;y-size;type|

Where:

\begin{description}
	\item[distance] distance from left edge of rail
	\item[x-size, y-size] dimension of a hole in mm
	\item[type] one of follows values:
	\begin{description}
		\item[0] hole
		\item[1] nut
		\item[2] screw
	\end{description}
\end{description}

Example: \verb|50;5.0;5.0;0;450;5.0;5.0;0|

\subsection{Drill type}

Information what types of drill holes is used in Clip Project: standard or special. It is only for unperforated rails. See also clipprj.drillHoles. Valid values: (numbers only).

\begin{description}
	\item[Name] clipprj.drillType
	\item[Scope] Rail
	\item[Valid values] 0-1 Numbers only
	\begin{description}
		\item[0] special drill holes
		\item[1] standard drill holes
	\end{description}
\end{description}

\subsection{End caps}

Article number of end caps for rail to be added. If not give, no end caps will be added for rail.

\begin{description}
	\item[Name] clipprj.endCap
	\item[Scope] Rail
\end{description}

\subsection{Bridge open/closed}

Is bridge open or close. Applicable only for some type of bridges.

\begin{description}
	\item[Name] clipprj.closed
	\item[Scope] Accessories
	\item[Valid values] 0-1 Numbers only
	\begin{description}
		\item[0] open
		\item[1] close
	\end{description}
\end{description}

\subsection{Coordinates}

Defines list of coordinates needed for positioning bridge or accessory on terminals. Best matching will be used for product data in CLIPPROJECT database.

\begin{description}
	\item[Name] clipprj.coordinates
	\item[Scope] Accessories
\end{description}

Two formats are allowed here:

\begin{enumerate}
	\item Defining points. Each coordinate use 2 fields separated by semicolon as: \verb|x-coord;y-coord|, where x-coord and y-coord are the position of the bridge or accessory on terminal in mm. Example: \verb|0.50;27.80|.
	\item Specifying level for bridges. Instead of giving position in mm a level identifier can be provided. Level name refers to CLIP PROJECT database and specific level must exist. Examples: \verb|O1|, \verb|O2|, \verb|M1|, \verb|M3|, \verb|U1|, etc.
\end{enumerate}

\subsection{ECAD Coordinates}

Specify level for bridges to set. Can be used as an alternative to clipprj.coordinates to defining bridge position.

\begin{description}
	\item[Name] clipprj.ECADcoordinates
	\item[Scope] Accessories
	\item[Valid values] (numbers only) 1-n
\end{description}

\subsection{Connected terminals ids}

List of terminals ids specified by clipprj.connectionId property of terminals. Can be used as alternative to reference mechanism to indicate connected terminals.\\

\begin{description}
	\item[Name] clipprj.connectionpoints
	\item[Scope] Accessories
\end{description}

Example: \verb|5;7| – connect 2 terminals with ids (clipprj.connectionId) 5 and 7.

\subsection{Consrtuction articles}

\begin{description}
	\item[Name] clipprj.constructionArticles
	\item[Scope] Accessories
\end{description}

List of additional article numbers needed for assembly module plug. Not accessible for all types of plugs.

Example:

\begin{lstlisting}[%
	label={code:xml_article_assembly},
	caption={Example of an article assembly},
	language=XML,
]
<o name="DPB Ex" type="clipprj.plug" id="9">
<pl>
	<p name="clipprj.articleNumber">3054040</p>
	<p name="clipprj.connectionId">6</p>
	<p name="clipprj.connectionpoints">5;7;8</p>
	<p name="clipprj.constructionArticles">3054079;3054105;</p>
	<p name="clipprj.coordinates">-2;</p>
	<p name="clipprj.id">7</p>
</pl>
\end{lstlisting}

Module plug in example above is made of 3 article numbers.

\subsection{Plug level position}

Specify plug position on the basis element (terminal), related to the top of
basis element or left side.

\begin{description}
	\item[Name] clipprj.levelPosition
	\item[Scope] Accessories
	\item[Valid values] (numbers only) 0-n, where n depends of type of element. Usually n is 0, 1, 2 or 3.
\end{description}

\subsection{Customer}

Customer name to be used only as a property of shopping cart.

\begin{description}
	\item[Name] clipprj.customer
	\item[Scope] Other
\end{description}

\chapter{Example project}

\lstinputlisting[%
	label={code:xml_example_project},
	caption={Example data structure of one rail},
	language=XML,
	%firstline=82,
	%firstnumber=82,
	%lastline=110
]{ECAD-Export-ExampleProject.xml}

The project contains one rail. On the rail there are two terminals and one End bracket. The End bracket has a reference \verb|<r>| to accessory. \verb|<r>| rid indicates in reference list to \verb|<ref>| with id equal $0$. Reference \verb|<ref>| maps $id=0$ to $oid=5$ where oid is \verb|<o>| id. Thus End bracket has one accessory (KLM 2).

Terminals has reference rid = 1 to one accessory. Reference maps id to object id ($oid=6$). Object with $id=6$ indicates bridge (FB 2-6). 

\chapter{Remarks}

Some attributes must use specific number of fields separated by semicolon. 
ECAD allows to use two equivalent notations of attributes e.g. A;B;C or A;B;C; - for three fields attributes.

It does mean that any character or characters after semicolon is treated like one field e.g. A;B; + \glqq{}space or any whitespace character\grqq{} is three fields attribute!  

% --------------------------------------------------------------
% Die Verzeichnisüberischten
\listoffigures
\listoftables
\lstlistoflistings

%\bibliographystyle{babplain}
%\bibliography{phoenix}

\end{document}
